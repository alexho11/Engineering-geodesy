\documentclass[12pt
,headinclude
,headsepline
,bibtotocnumbered
]{scrartcl}
\usepackage[paper=a4paper,left=25mm,right=25mm,top=25mm,bottom=25mm]{geometry} 
\usepackage[utf8]{inputenc}
\usepackage[ngerman]{babel}
\usepackage{graphicx}
\usepackage{multirow}
\usepackage{pdfpages}
%\usepackage{wrapfig}
\usepackage{placeins}
\usepackage{float}
\usepackage{flafter}
\usepackage{mathtools}
\usepackage{hyperref}
\usepackage{epstopdf}
\usepackage[miktex]{gnuplottex}
\usepackage[T1]{fontenc}
\usepackage{mhchem}
\usepackage{fancyhdr}
%\setlength{\mathindent}{0pt}
\usepackage{amssymb}
\usepackage[list=true, font=large, labelfont=bf, 
labelformat=brace, position=top]{subcaption}
\setlength{\parindent}{0mm}

\setlength{\parindent}{0mm}

\pagestyle{fancy}
\fancyhf{}
\lhead{Ingenieurgeodäsie I\\Übung 2: Toleranzen und Standardabweichungen}
\rhead{Hsin-Feng Ho \\3378849}
\rfoot{Seite \thepage}
\begin{document}
\section{Toleranzen und Standardabweichungen}
Diese Übung geht es um die Toleranzen und Standardabweichungen. Dafür ist die Bestimmung der Genauigkeit der Richtung- und Streckenmessung notwendig. 
Darüber hinaus wird ein geeignetes Gerät ausgewählt, um die Messung durchzuführen.
\subsection{Benötigte Genauigkeit}
Da nur ein Teil der gesamten Maßtoleranz als Vermessungstoleranz zugelassen ist,
wird zunächst mal die Vermessungstoleranz bestimmt.
\begin{align*}
    T_M=T\cdot\sqrt{2p-p^2}&=10\sqrt{2\cdot\frac{1}{3}-\frac{1}{3^2}}\\
    &=7.45\,\mathrm{mm}
\end{align*}
Aus die Maßtoleranz kann man die erforderliche Standardabweichung berechnen.
\begin{align*}
    \sigma=\dfrac{T_M}{2k}=\dfrac{7.45}{2\cdot 1.96}=1.9\,\mathrm{mm}
\end{align*}
\subsection{Berechnung der Standardabweichung der Richtungs- und Streckenmessung}
Um die Koordinaten des Punktes S zu bestimmen, werden vom Punkt A Strecke und   
\end{document}